\documentclass{beamer}
\usepackage[utf8]{inputenc}
\usepackage{amsmath, amssymb} % For mathematical symbols
\usepackage{graphicx} % For including images if needed
\usepackage{booktabs} % For nicer tables if needed

\title{Assessing the Impact of Estimating Capital Market Correlation Structures on Risk Parity Strategies}
\author{Rodrigo G Giannotti}
\institute{MPE - EESP FGV}
\date{\today}

\begin{document}

\begin{frame}
    \titlepage
\end{frame}

\begin{frame}{Outline}
    \tableofcontents
\end{frame}

% Section 1: Introduction
\section{Introduction}
\subsection{Evolution in Portfolio Theory}
\begin{frame}{Introduction - Evolution in Portfolio Theory}
    \begin{itemize}
        \item 1952 \textbf{Markowitz Modern Portfolio Theory} – Introduced mean-variance optimization, forming the basis of portfolio selection.
        
        \item 1952 - 1960s \textbf{Extensions of MPT} – Developing techniques incorporating risk factors, improving on traditional diversification.
        
        \item 1970s, 80s, 90s \textbf{Post-Modern Portfolio Theory} – Shift toward balancing risk contributions, emphasizing equal risk weights over equal capital allocation.
        
        \item 2000s - Today \textbf{Modern Approaches for Covariance Estimation} – Application of advanced ML methods to improve covariance matrix estimation in always evolving risk-parity-inspired modern frameworks.
    \end{itemize}
\end{frame}

\subsection{Motivation and Research Question}
\begin{frame}{Introduction - Motivation and Research Question}
    \begin{itemize}
        \item \textbf{Motivation} – How relevant is market structure estimation for post-modern portfolio management strategies?
        \item \textbf{Research Question} – Can we measure how the accuracy of estimation of correlation structures impacts risk-parity strategies?
    \end{itemize}
\end{frame}

% Section 2: Literature Review
\section{Literature Review}
\subsection{Risk Parity Strategies}
\begin{frame}{Literature Review - Risk Parity Strategies}
    \begin{itemize}
        \item 2005 - \textbf{First RP fund} - Formalizing the concept as a real-life portfolio building strategy. %PanAgora became the first firm to openly state the use of risk parity portfolios and to promote a product that was based on it.
        \item 2010 - \textbf{Housing Crisis Aftermath} - Good results of RP strategies during the 2008 crisis bring in attention. %AQR and Callan Investments, 2 big and renowned firms publish white papers explaining in-depth RP to investors. 
        \item 2011 - Today - \textbf{Advancements and Refining} - New proposals to further enhance the potential results of such strategies.
        \begin{itemize}
            \item 2015 - \textbf{Hierarchical Risk Parity} - M. L. de Prado
            \item 2022 - \textbf{Kurtosis-based Risk Parity} -  M. D. Braga et al.
        \end{itemize}
    \end{itemize}
\end{frame}

\subsection{Covariance Matrix Estimation}
\begin{frame}{Literature Review - Covariance Matrix Estimation}
    The numerical stability of the covariance matrix has been under scrutiny since the early days of modern portfolio theory.
    \begin{itemize}
        \item 1980 - \textbf{Jobson, Korkie} - Estimation for Markowitz eficient portfolios.
    \end{itemize}
    Some have proposed employing linear algebra tools to counteract it.
    \begin{itemize}
        \item 2003 - \textbf{Ledoit, Wolf} - Using Shrinkage to improve stability. 
    \end{itemize}
    Others have proposed using Machine Learning to find better shrinkage methods.
    \begin{itemize}
        \item 2022 - \textbf{Lu, Ndiaye, Simaan} - Using Reinforcement Learning to better shrink the covariance matrix.
    \end{itemize}
\end{frame}

\subsection{Backtest Robustness Evaluation}
\begin{frame}{Literature Review - Backtest Robustness Evaluation}
%The field of Portfolio Management heavily relies on observing the past and evaluating hypothetical portfolios over past periods to evaluate the techniques used to find such portfolios. These tests of theoretical past portfolios are called backtests.
%Many metrics have been proposed to assess the performance of a given backtest. More recently the statistical significance of these metrics has also begun to be a point of interest in academic literature. Some pivoting papers have been:
    \begin{itemize}
        \item Metrics to evaluate.
        \begin{itemize}
            \item   1952 - \textbf{Markowitz} - Beginnings of Modern Portfolio Theory.
            \item   1964 - \textbf{Sharpe} - Reward-to-variability ratio.
            \item   1973 - \textbf{Treynor, Black} - Reward-to-market correlation ratio.
        \end{itemize}
        \item Assessing the statistical validity of such measures.
        \begin{itemize}
            \item   2014 - \textbf{Bailey, Prado} - Adjusting Sharpe ratio for non-normality.
            \item   2019 - \textbf{Paulsen, Sohl} - Using the reward-to-variability concept to propose an information criterion for strategy selection.
        \end{itemize}
    \end{itemize}
\end{frame}

% Section 3: Methodology
\section{Methodology}
\subsection{Evaluating Accuracy}
\begin{frame}{Methodology - Evaluating Accuracy}
    \begin{itemize}
        \item   Probability Density Function Fitness.
        \item   Forecast vs Realized Market Volatility.
        \item   Forecast vs Implied Market Volatility.
    \end{itemize}
\end{frame}

\subsection{Evaluating Strategy Performance}
\begin{frame}{Methodology - Evaluating Strategy Performance}
    \begin{itemize}
        \item   Risk-Adjusted and Beta-Adjusted Returns.
        \begin{itemize}
            \item Realized vs Benchmark Portfolios.
            \item Realized vs Expected.
        \end{itemize}
        \item   Forecast vs Realized Portfolio Volatility.
        \item   Descriptive Analysis of Return Attribution.
    \end{itemize}
\end{frame}

% Section 4: Expected Findings
\section{Expected Findings}
\begin{frame}{Expected Findings}
We hope to have data that allows us to answer:
    \begin{itemize}
        \item Do better estimations lead to better portfolios?
        \item Do better estimations lead to more balanced portfolios?
        \item Are better estimations better than more robust estimations for both?
        \item Are there market circumstances where it is more (or less) relevant to have better covariance estimators?
    \end{itemize}
\end{frame}

% Section 5: Deliverables Timeline
\section{Deliverables Timeline}
\begin{frame}{Deliverables Timeline - Nov 2024 - Feb 2025}
    \begin{itemize}
        \item Nov 2024 
        \begin{itemize}
            \item \textbf{Scope Definition} % Define which assets will be in our study universe in which time windows.
            \item \textbf{Data Acquisition Start}
        \end{itemize}
        \item Dec 2024
        \begin{itemize}
                \item \textbf{Data Acquisition Finish}
                \item \textbf{Covariance Matrix Algorithms} % Implement algorithms to estimate the covariance matrix for all the chosen methods.
                \item \textbf{Backtest Evaluation Algorithm} % Implement algorithms to evaluate the desired backtest metrics for a given portfolio. 
        \end{itemize}
        \item Jan 2025
        \begin{itemize}
            \item \textbf{Tune Code to the first batch of tests} % Evaluate first findings, ensure that the code structure is working, all necessary metrics are reported correctly and next batches can be run autonomously.
            \item \textbf{First draft of the literature revision}
        \end{itemize}
        \item Feb 2025
            \begin{itemize}
                \item \textbf{Results tables and test conclusions}
                \item \textbf{First draft of results}
            \end{itemize}
    \end{itemize}
\end{frame}

\begin{frame}{Deliverables Timeline - Mar 2025 - Jun 2025}
    \begin{itemize}
        \item Mar 2025
            \begin{itemize}
                \item \textbf{First complete draft}
            \end{itemize}
        \item Apr 2025
            \begin{itemize}
                \item \textbf{Fine tune conclusions and text}
                \item \textbf{First presentation draft}
            \end{itemize}
        \item May 2025
            \begin{itemize}
                \item \textbf{Text final version}
                \item \textbf{Presentation rehearsals}
            \end{itemize}
        \item Jun 2025
            \begin{itemize}
                \item \textbf{Presentation}
            \end{itemize}
    \end{itemize}
\end{frame}

\end{document}
